\input{/Users/AlexHahn/LatexReusableFiles/PhysicsHeader.tex}
\begin{document}
\title{Summary of : Realization of Deterministic Quantum Teleportation with
Solid State Qubits}
\author{Alex Hahn}
\date{\today}
\maketitle

\subsection*{Introduction}
As quantum computational machines come to fruition there develops a need to
analyse, apply, and take advantage of the peculiar behavior of quantum
mechanics. One way, which we discussed somewhat at the end of our course, is
communication via entangled states, otherwise known as quantum
teleportation. I've chosen to write about this paper from the Physics
Department in Zurich Switzerland$^*$ (see citation at end) because it has a more
graspable and relatable presentation relative to our class when compared to the
more physics based (``what happens to the Hamiltonian", ``Ising model", etc) presentations of other papers. However there are still many sections that go slightly above my
head / beyond the scope of the course which I will also briefly point out.\\

The paper starts off with an abstract on the relation between a need to transfer
information in the classical sense and a claim about its realization in the
quantum communication and information processing world. The paper claims to
detail
a ``superconducting" quantum circuit capable of the teleportation of the
state of a qubit to a distant observer (``macroscopic"). This would be an
important result in making ``repeaters" for quantum communication.\\

The paper is not overly cavalier and points out despite promises of
advancement in information science, logic gate circuitry, construction of
entangled states, algorithmic speed up, and error correction, the physical
engineering of networks and ``connecting topology" on actual chips is a
central problem.

\subsection*{Implementation, Protocol, and Teleportation}

The paper then moves onto an explanation of superconducting circuit
architecture and its exploitation which I will leave out because a) I don't
understand it fully and b) it is way out of the scope of the class. The
punchline seems to be qubit-qubit coupling on the hardware (``accross
horizontal and vertical resonators").\\

The physical experiment's design is to (at some high rate) quantum teleport over
a macroscopic distance of 6 mm between to quantum systems. An important
coursework related point is then made that the experiment succeed with
``order unit probability" for any input state due to the fact that they can
prepare maximally entangled two-qubit states and distinguish all four
two-qubit states from observation via measurement.\\

as detailed in class, one of the four Bell states:
$$\ket{\Phi^\pm}=(\ket{00}\pm\ket{11})/\sqrt{2}\text{ and }
\ket{\Phi^\pm}=(\ket{01}\pm\ket{10})/\sqrt{2}$$

At the risk of going over stuff we've explained proficiently in class, the
protocol is that an unknown state $\ket{\psi_{in}}$ of qubit ``Q1" which the sender has is transferred to to the
receiver's qubit ``Q3". To do this both parties have already prepared an
entangled Bell state between another qubit Q2 and Q3. When the sender
measures Q1 and Q2 the qubits in his possesion are put into one of the bell
states enumerated above. More importantly the receiver's Q3 is projected into
a state
$$\ket{\phi_{out}}=\{\textbf{1},\hat\sigma_x,\hat\sigma_z,i\hat\sigma_y\}\ket{\psi_{in}}$$

Which signifies that the he holds the input state but with a single
rotation applied. Lastly the ``feed forward" (?) step is performed (think
opposite of ``feedback" I guess).\\

We finally arrive at a point where we can detail and model the operations/ protocol via gates we
learned about in class!

\begin{center}
\includegraphics[scale=.5]{/Users/AlexHahn/Desktop/circuit1.png}
\end{center}

where  we first prepare a Bell state of Q2 and Q3 in the blue box, the
arbitrary $\ket{\psi_{in}}$ state is in green, and the measurement (of the Bell
state) occurs in the red box. The $H$ operators is our familiar
Hadamard while the $X$ and $Z$ operators are Pauli matrices $\hat\sigma_x$ and
$\hat\sigma_z$, and lastly we have our familiar CNOTE-gates.\\

However for the paper's implementation of the circuit, ``controlled-PHASE"
gates are used (vertical lines with $\bullet$'s on both sides)
and individual Qbit rotations $R_{\pm y}^\theta$ of angle $\theta$ about the
$\pm y$ axis $^*$ (as detailed by the diagrams in the paper)

\begin{center}
\includegraphics[scale=.4]{/Users/AlexHahn/Desktop/circuit2.png}
\end{center}

The paper then proceeds to detail probabilistic arguments ($50$ (Feynman
history in our book)$,66,87\%$) of
success then rectifies these discrepancies via ``photonic
continuous-variable states"(?) and a final ''conditional" rotation.
(Followed by some advanced discussion of ions, and traps)

\subsection*{Advanced section on hardware and physical set up (a fair
amount of the paper is dedicated to this)}

Not sure what it means but ``three superconducting transmons" were used
along with (coupled to) ``three superconducting coplanar waveguide
resonators". If your interested and a diagram would suffice I've attached one
in the appendix at the end.

\subsection*{Results}
The experiment was a success with results verifying the previous claims I've written about and is summarized neatly in this graphic:

\begin{center}
\includegraphics[scale=.4]{/Users/AlexHahn/Desktop/results.png}
\end{center}



\section*{Source}

L.Steffen, A.Fedorov, M.Oppliger, Y.Salathe, P. Kurpiers, M. Baur, G.
Puebla-Hellmann, C. Eichler, and A. Wallraff.`` Realization of Deterministic Quantum Teleportation with Solid State Qubits". \textit{ Department of Physics, ETH Zurich}, Switzerland Feb 25, 2013\\

\texttt{http://arxiv.org/pdf/1302.5621v1.pdf}\\



Implementation of hardware:
\begin{center}
\includegraphics[scale=.4]{/Users/AlexHahn/Desktop/confusinf.png}
\end{center}













\end{document}
