\input{/Users/AlexHahn/LatexReusableFiles/PhysicsHeader.tex}
\setlength{\parindent}{0pt}
\begin{document}
\title{Physics 7681, Problem Set 2}
\author{Alex Hahn}
\date{\today}
\maketitle

\textbf{Problem 1:} Generalized Born Rule I\\

In class we covered section 1.9 of the course text, which showed that
successive measurements of Qbits gives results with same probabilities as
measuring all at once. Let's see how this works explicitly in a simple case by
considering the most general 3-Qbit state

$$\ket{\Psi}=\sum_{x=0}^7\alpha_x\ket{x}_3=\alpha_{000}\ket{0}\ket{0}\ket{0}+\alpha_{001}\ket{0}\ket{0}\ket{1}+\alpha_{010}\ket{0}\ket{1}\ket{0}+\alpha_{011}\ket{0}\ket{1}\ket{1}+\alpha_{100}\ket{1}\ket{0}\ket{0}+\alpha_{101}\ket{1}\ket{0}\ket{1}+\alpha_{110}\ket{1}\ket{1}\ket{0}+\alpha_{111}\ket{1}\ket{1}\ket{1},$$
(where $\sum_x|\alpha_x|^2=1$). The 3 Qbits are labelled from the left: Qbit
2, Qbit 1, Qbit 0. For Qbits prepared in the above state, what is the
probability of the measurement ($x$), and in what state are the Qbits left
after the measurement, for each of the following cases:\\

a) The 3 Qbits 2,1,0 are measured simultaneously to be (1)\\

This is simply the amplitude of the corresponding state in the superposition
of states, from our our this is simply $|\alpha_{101}|^2$. Since we have
collapsed (reduced the state) so it must be in the state $\ket{101}$ after
measurement\\

b) Qbit 2 is measured to be (1)\\

We have several terms in the superposition that hold some $alpha$ amplitude
(probability) that the second Qbit will be 1. The corresponding probability is
the sum of the probability of these states (assuming our Wave function has
already been normalized):
$|\alpha_{100}|^2+|\alpha_{101}|^2+|\alpha_{110}|^2+|\alpha_{111}|^2$. The
state of the Qbits left after the measurement is a product state,
$\ket{1}_1\ket{\Phi_1}_2$, where $\ket{\Phi_1}$ is normalized (but once again not
necessarily orthogonal)\\

c) Qbits 2,1 are measured simultaneously to be (1,0)\\

This corresponds to only two terms (states) in our superposition of states
with sum probability $|\alpha_{100}|^2+|\alpha_{101}|^2$. To find the what
state the Qbits are left in requires applying the most general form of Borns
rule with $\ket{\Psi}_{2+1}=\sum^{2^2}_{x=0}\alpha_x\ket{x}_m\ket{\Phi_x}_1$
which after measurement the state of all $m+n$ (3) Qbits will be the product
state $\ket{10}_2\ket{\Phi_{10}}_1$ (the $m$ measured Qbits are in state
$\ket{10}_2$ and the $n$ unmeasured ones are in state $\ket{\Phi_{10}}_1$.\\

d) First Qbit 2 is measured to be (1), then Qbit 1 is measured to be (0)\\

We could apply Born's rule with $m=2$ and $n=1$ but since we know that
successive measurements are equivalent to simultaneous ones (probability is
conserved) we know the probability is equivalent to our previous example,
using Born's rule we could explicitly write the resulting state as $\ket{1}_1\ket{0}_1\ket{\Phi_{10}}_1$
to make explicit that two measurements were made consecutively.
(we could even separate the into two different $\ket{\Phi}$'s if we were to
truly apply Born's rule each time).\\

e) First Qbits 2,1 are measured to be (1,0), then Qbit 0 is measured to be (1)\\

Similar to d) we know that this is the same probability as a). We could write
this out as 2 applications of Born's rule with the resulting $\ket{\Phi}_0$ state being a delta function (it's only possible for it to have the observed state
after measured). Applying Born's rule gives $\ket{10}_2\ket{1}_1\ket{\Phi_{101}}_0$ (once again combining the $\Phi$'s instead of writing out twice), but its really just the observed state itself,
$\ket{101}$\\

\textbf{Problem 2:}
The most powerful version of the Born rule is in the form stated at the end of
section
1.9 (eqs. (1.80)–(1.81)). The rule notes that the general state of $m+n$ Qbits
can be written
$$\ket{\Psi}_{m+n}=\sum_{0\leq x<2^m}\alpha_x\ket{x}_m\ket{\Phi_x}_n,$$
where $\sum_x|\alpha_x|^2=1$ and states $\ket{\Phi_x}_n$ are normalized, but
not necessarily orthogonal. The rule asserts that if only the $m$ Qbits
associated with the states $\ket{x}_m$ above are measured, then with
probability $|\alpha_x|^2$ the result will be $x$, and after the measurement
the state of all $m+n$ Qbits will be the product state
$$\ket{x}_m\ket{\Phi_x}_n,$$
in which the $m$ measured Qbits are in the state $\ket{x}_m$ and the $n$
unmeasured ones are in the state $\ket{\Phi_x}_n$. The following confirms that
this rule satisfies the reasonable requirement that measuring $r$ Qbits and
then immediately measuring $s$ more, is exactly the same as measuring the
$r_s$ Qbits all at once:\\


(a) Given $n$ Qbits in the state $\ket{\Phi}_n$, suppose $r$ of the Qbits are
measured, immediately after which $s$ additional Qbits are measured (with
$r+s<n$). Find the possible states of the $n$ Qbits after the two measurements
and their associated probabilities, by first applying the rule to the $r-$Qbit
measurement and then applying the rule a second time to the subsequent
$s$-Qbit measurement.\\

After the first measurement, using the generalized Born rule, after the first
measurement of $r$ Qbits we are left with the product state

$$\ket{x}_r\ket{\Phi_x}_{n-r}$$

Now we apply the generalized Born rule again to this state but this time now
measuring $s$ Qbits

$$\ket{y}_s\ket{x}_r\ket{\Phi_x}_{n-r}\ket{\Phi_y}_{n-(r+s)}$$

From page 29 we know that the $\ket{\Phi}$'s carry $\alpha$ probabilities that
multiply the inverse phase factor. We then can compute that after the first
$r$ Qbit measurement we get a $\frac{1}{\sqrt{p(x)}}$ term, and after the $s$ Qbit
measurement we get a factor of $\frac{1}{\sqrt{p(y)}}$. Putting this all together we
have

$$\ket{y}_s\ket{x}_r\ket{\Phi_x}_{n-r}\ket{\Phi_y}_{n-(r+s)}=\ket{y}_s\ket{x}_r\ket{\Phi_xy}_{n-(r+s)}=\ket{y}_s\ket{x}_r\frac{1}{\sqrt{p(y)}}\frac{1}{\sqrt{p(x)}}\sum_x\alpha\ket{x^*}_{n-(r+s)}$$

where $x^*$ is it's own distinct state\\

(b) Show that the result of (a) is the same as would be found, more simply by
just applying the rule a single time to a single measurement of all $r+s$
Qbits\\

Applying the generalized born rule we would have

$$\ket{xy}\ket{\Phi_{xy}}_{n-(s+r)}$$

and in terms of the probabilities we see it result in the same as in part (a)
because

$$\frac{1}{\sqrt{p(x)}}\frac{1}{\sqrt{p(y)}}=\frac{1}{\sqrt{p(xy)}}$$

The RHS being our single $r+s$ measurement. We'd overall have

$$\ket{xy}_{r+s}\frac{1}{\sqrt{p(xy)}}\sum_x\alpha\ket{x^*}_{n-(r+s)}$$

In classical physics this is analogous to how throwing a dice twice
(consecutively) is the same as throwing 2 identical dice at once.\\

(c) We can also write the above state in the full basis of $m+n$ Qbits as
$\ket{\Psi}_{m+n}=\sum_{0\leq x'<2^{m+n}}\gamma_{x'}\ket{x'}$. Write down an
expression for the $\alpha_x$ in terms of the amplitudes $\gamma_{x'}$.\\

I believe that the physical interpretation of this is that we are writing the wave function (superposition of states)
$\ket{\Psi}_{m+n}$ in a basis that is more directly related to the post $m$ Qbit
measurement ie the still  undeterministic (not collapsed) $\ket{\Phi}$ function (superposition of states). If that makes since then these are just ``updated" $alpha$ coefficients. I think you might be able to get them by, once you observe the $m$ Qbits, by using the normalization condition on the resulting product state. This would mean the original $\alpha$ coefficients would be 1/$\gamma(x)$ for each corresponding coefficient\\


\textbf{Problem 3:} Indirect measurement of Qbit amplitudes?\\

a) Suppose the upper Qbit starts in the state $\alpha\ket{0}+\beta\ket{1}$ anf
the lower Qbit starts in the state $\ket{0}$, as depicted above. Can some long
sequence of cNOT gates followed by measurement gates give some statistical
sampling of the state of the upper Qbit, i.e. canthe above arrangement with
some largeg number $N$ of cNOT/ measurement gate pairs be used to estimate the
values of $|\alpha|^2$ and $|\beta|^2$ without collapsing the Qbit state?
Characterize
the outputs of the measuring devices and the final states above after some
number of
measurements.\\

This seems to violate the no cloning theorem: because we collapsed some wave function by measurement we can't simply restore it to its original state by applying an
entangling operator to it (in this case cNOT) then perform a measurement
again and repeat the process many times. Just by re-entangling the 2
Qbits we don't guarantee that the probabilistic amplitudes are the same in the
entire system each time. Only if the upper Qbit was in a pure state could accurate
statistical inferences be made (ie either almost all probability $\ket{0}$ or
all probability $\ket{1}$. The final state will just be the 0 or 1 state
observed after the first measurement flipped $N$ (or $N-1$) times (or
following the straightforward operations on the $\ket{0}$ or $\ket{1}$ state.\\

If you want to get into the quantum parallelism (even multi-worlds argument)
you'd argue that one you make that first measurement you've determined what
``world" you were in, you couldn't take advantage of more calculation(s) done
in parallel universes). This is also analogous to a classical example of two
billiard balls with a know Hamiltonian describing the system hitting each
other then separating light years apart. If you observe one and measure its
properties, you can't re-hit it (or do whatever to it) and take more
measurements, then do the same over and over to sketch out some probabilistic  wave function of the other
ball... once you observed it once, you determined the other ball's state\\

b) Even though the part a) doesn't correctly sample indeterministic quatum
states adding the additional Qbit I think actually makes things worse. I think
this because in the diagram 3 CNOT operators are applied before any
measurement is made. The first Qbit might be more affected by this more
complicated system due to interactions with the ``program" and our initial
measurement might not reflect anything about the original Qbit, it might in
theory be the same as part a) though\\

\textbf{Problem 4} Classical light and quantum photon polarizations\\
The polatization of either a single photon or classical light (lots of
photons) is described by a two dimensional unit vector in the plane transverse
to its direction of motion. Consider photons(s) polarized at an angle $\theta$
from the horizontal ($0\leq\theta<\pi$ since we're not concerned with an
overall phase). Let $P_\theta$ denote a polarizer oriented at angle $\theta$
from the horizontal (where $P_h\equiv P_0$ and $P_v\equiv P_{\pi/2}$ denote
horizontal and vertical polarizers, respectively). In classical
electrodynamics, $h$-polarized light with intensity $I_0$ incident on
$P_\theta$ emerges with polarization $\theta$ and intensity $I_0\cos
^2\theta$. In quantum mechanics, each individual photon in the light beam
starts $h$-polarized and has a probability $\cos^2\theta$ of transmission
(otherwise absorbed), resulting in the observed wave behavior for large
numbers of photons.\\

Let $R_\theta=\mat{\cos\theta & -\sin\theta \\ \sin\theta & \cos\theta}$ and
$Z=\mat{1 & 0\\ 0 & -1}$ act, respectively, as rotation of the polarization
plane, and reflection of the vertical direction (the action of a half-wave
plate that flips the vertical component of polarization, i.e., reflects about
the horizontal axis). Note that $ZR_\theta Z=R_\theta$.\\

a) Show that $F_\theta\equiv R_\theta Z R_{-\theta}$ reflects about an axis at
angle $\theta$ clockwise from the horizontal (half-wave plate rotated by
$\theta$), and confirm that $H=F_{\pi/8}$ (Hadamard is realized by a half-wave
plate at an angle of $22.5^\circ$). Now consider a light beam polarized at
$\theta=\pi/4$. According to the above the above formula, half the intensity
would transmit through $P_h$, and half through $P_v$. It is tempting to
imagine that $\pi/4$-polarized light consists of equiprobability parts $h$ and
$v$ polarized light. To assess that possibility, consider placing a half-wave
plate $H=F_{\pi/8}$ before a $P_v$. What fraction of the intensity of initially $h$-polarized light would transmir $I_h\rightarrow F_{\pi/8}\rightarrow P_v\rightarrow ?$, and what fraction of initially $v$-polarized light $I_v\rightarrow F_{\pi/8}\rightarrow P_v\rightarrow ?$ Compare to the fraction of $\pi/4$-polarized light that transmits $I_{\pi/4}\rightarrow F_{\pi/8}\rightarrow P_v\rightarrow$ ? to infer that the latter can't consist of equiprobability parts $h$ and $v$. (Note this is the same argument as after eq. 1.72) in the book for a single photon.)\\

I believe that there are 2 ways to show this. The purely mathematical way
would be to show that $F_\theta=R_\theta Z R_{-\theta}$ maps the unit basis
vectors to a reflection at an angle $\theta$ clockwise from the horizontal,
since we are given all the matrices this is quite computationally possible but a
little tedious. Another way to show it is to take advantage of the given
identity $ZR_\theta Z=R_{-\theta}$. This would mean

$$F_\theta\equiv R_\theta Z R_{-\theta}=R_\theta ZZ R_\theta Z = R_\theta
R_\theta Z$$
since we perform two equal rotations it is clear / straightforward to
determine the line that we equivalently reflected over ($\theta$ degrees from
the horizontal)\\

For the second part of the first question First we multiply out the actual
matrices $R_\theta Z R_{-\theta}=F_\theta$ to get
$$F_\theta=\mat{\cos^2\theta-\sin^2\theta & 2\sin\theta\cos\theta\\
2\sin\theta\cos\theta & \sin^2\theta-\cos^2\theta}$$

solving for $\theta=\pi/8$ we get

$$\mat{\frac{1}{\sqrt{2}} & \frac{1}{\sqrt{2}}\\ \frac{1}{\sqrt{2}} &
-\frac{1}{\sqrt{2}}}$$

which is the Hadamard $H$ operator.\\

For the last part (three parts) of the question: the fraction of initially
$h$-polarized light that transmits when a half wave plate $H$ is placed before a
$P_v$ is $\cos^2(\pi2+2\pi/8)=.5$, similarly the fraction of $v$-polarized
light that transmits is $\cos^2(2\pi/8)=.5$. However the fraction of
$\frac{\pi}{4}$-polarized light that transmits is $\cos^2(\pi/4+2\pi/8)=1$\\

b) (i) Now consider the effect of $F_{\pi/6}Z$ on a polarization state, e.g.,
with what polarization does an $h$ polarized photon emerge, $h\rightarrow Z\rightarrow F_{\pi/6}\rightarrow ?,$ and similarly for a $\pi/6$-polarized photon, $\pi/6\rightarrow Z \rightarrow F_{\pi/6}\rightarrow ?$ \\

$h$-polarized $\rightarrow ~\frac{\pi}{3}$, $\frac{\pi}{6}$-polarized
$\rightarrow~ \frac{\pi}{2}$\\


(ii) Let $\phi\equiv \frac{1}{2n+1}(\pi/2)$ and suppose a photon starts $\phi$-polarized. Via a sequence of mirrors, $F_\phi Z$ is applied $n$ times, $\phi\rightarrow(Z\rightarrow F_\phi)^n\rightarrow?$ What is the probability it measures to have vertical polarization?\\

If $n=1$ it must have vertical polarization (probability 1) otherwise it can't
be vertically polarized ($n>1$)\\

c) An $h$-polarized photon has zero probability of passing through a vertical
polarizer $P_v$. Consider placing a diagonal polarizer $P_{\pi/4}$ in front of
$P_v$.\\

(i) What is the probability of passing through both,
$h\rightarrow P_{\pi/4}\rightarrow P_v\rightarrow ?$

This is equivalent to flipping a coin twice and getting two head
(probabilistically): $\frac{1}{2}\frac{1}{2}=\frac{1}{4}$\\

(ii) Now suppose $N-1$ polarizers, successively rotated by $\pi/2N$, are
inserted before $P_v$, what is the probability of passing through all $N$,
$h\rightarrow P_{\pi/2N}\rightarrow P_{2\pi/2N}\rightarrow
...\P_{j\pi/2N}...\rightarrow P_{(N-1)\pi/2N}\rightarrow P_v\rightarrow?$, in
the limit of large $N$?\\

In the limit of large $N$ the probability goes to 1 (eventually must pass
through each next polarizer as $N\rightarrow\infty$).\\

\textbf{Problem 5:} 2-Qbit to 1-Qbit functions\\

Attached at back\\



\textbf{Problem 6:} Another generalization of Deutch's problem
Deutsch’s problem involved a 1-bit to 1-bit function $f$ which was either
constant or
not. The Deutsch-Jozsa problem generalizes this to an $n$-bit to 1-bit function
which is
either constant (0 on all inputs or 1 on all inputs) or “balanced” (returns 1
for half of the
input domain and 0 for the other half). The task is to determine whether $f$ is constant or
balanced.\\

a) For a classical computer, how many evaluations of $f$ are required in the
worst and
best cases? What are the probabilities of the worst and best cases? (assume an
equal prior
probability of $f$ being constant or balanced)\\

The worst case occurs when we see half the evaluations as the same, then an
additional evaluation to determine whether $f$ was balanced or not. This would
look like $2^{n-1}+1$ $f$ evaluations. The worst case probability is
$.5+.5((\frac{n}{2}-1)!\cdot
\frac{(\frac{n}{2}-1)}{(n-1)!}=.5+\frac{.5}{(n-1)!}$.

In the best case we simply need to check two evaluations, if they are not the
same $f$ is balanced. The probability of the best case looks like
$\frac{2^{n-1}}{2^n-1}$

b) (some ideas from wiki)\\

Apply the Hadamard transformation to obtain
$$\frac{1}{\sqrt{2^{n+1}}}\sum^{2^n-1}_{x=0}\ket{x}(\ket{0}-\ket{1})$$

applying the $U_f$

$$\frac{1}{\sqrt{2^{n+1}}}\sum^{2^n-1}_{x=0}\ket{x}(\ket{f(x)}-\ket{1\oplus
f(x)})$$

For each $x$ $f(x)$ is either 0 or 1 computing

$$\frac{1}{\sqrt{2^{n+1}}}\sum^{2^n-1}_{x=0}(-1)^{f(x)}\ket{x}(\ket{0}-\ket{1})$$

Applying the right side Hadamard transformation to each Qbit

$$\frac{1}{2^n}\sum^{2^n-1}_{x=0})(-1)^{f(x)}\sum^{2^n-1}_{y=0}(-1^{xy}\ket{y}=\frac{1}{2^n}\sum^{2^n-1}_{y=0}\left[\sum^{2^n-1}_{x=0}(-1)^{f(x)}(-1)^{xy}\right]\ket{y}$$

where $xy$ is the sum of the bitwise product ($x_0y_0\oplus x_1y_1\oplus
...\oplus x_{n-1}y_{n-1}$).\\

Examining the probability of measuring $\ket{0}^{\otimes n}$

$$\left|\frac{1}{2^n}\sum^{2^n-1}_{x=0}(-1)^{f(x)}\right|$$

which evaluates to 1 if $f(x)$ is constant, and to 0 if $f(x)$ is balanced.\\





\textbf{Problem 7:} : Clauser-Horne-Shimony-Holt (’69) / Bell (’64)
Inequality, Part II
Recall the game played by Alice and Bob in problem 5 of the previous set:
they receive
random bits $x$, $y$, then (without communicating) output bits $a$, $b$, and win if $a\oplus b =xy$. We saw that no (local deterministic) strategy permitted them to win with greater than
probability 3/4, i.e., no possible local hidden variables permit Alice and Bob
to produce
a, b systematically from $x$, $y$, such that $a \oplus b=xy$ more than 3/4 of
the time.

Now Now suppose Alice and Bob share the entangled state
$\frac{1}{\sqrt{2}}(\ket{00}+\ket{11})$, with Alice
holding one Qbit and Bob holding the other Qbit. Suppose they use the
following strategy:
if $x = 1$, Alice applies the unitary matrix $R_{\pi/6}=\mat{\cos
\frac{\pi}{6} & -\sin\frac{\pi}{6}\\ \sin\frac{\pi}{6} & \cos\frac{\pi}{6}}$ to
her Qbit, otherwise
doesn't, then measures in the standard basis and outputs the result as $a$. If $y =
1$, Bob
applies the unitary matrix$R_{-\pi/6}=\mat{\cos
 24 \frac{\pi}{6} & \sin\frac{\pi}{6}\\ -\sin\frac{\pi}{6} & \cos\frac{\pi}{6}}$ to his Qbit, otherwise  doesn’t, then
 measures in the standard basis and outputs the result as $b$. (Note that if the
 Qbits were
 encoded in photon polarization states, this would be equivalent to Alice and
 Bob rotating
 measurement devices by$\pi/6$ in inverse directions before measuring.)
 Using this strategy:\\

 a) Show that if $x=y=0$, then Alice and Bob win the game with probability
 1.\\

 Let the superposition of Alice's Qbit be $\alpha_0\ket{0}+\alpha_1\ket{1}$
 and bobs be $\beta_0\ket{0}+\beta_1\ket{1}$. The entangled 2 Qbit state is
 then fully described by
 $\alpha_0\beta_0\ket{00}+\alpha_0\beta_1\ket{01}+\alpha_1\beta_0\ket{10}+\alpha_1\beta_1\ket{11}.$\\

 In this case once Alice or Bob observe (measure) their Qbit, the other is
 forced to take it's value, so even though they each have a probability of 1/2
 for either a 0 or a 1, they end up returning the same measurement due to the entanglement every time $P=1$\\

b) Show that if $x=1$ and $y=0$ (or vice versa), then Alice and Bob win with
probability 3/4\\

Applying the agreed upon unitary operation

$$R_{\pi/6}\mat{\alpha_0\\
\alpha_1}=\mat{\frac{\alpha_0\sqrt{3}}{2}-\frac{\alpha_1}{2}\\
\frac{\alpha_1\sqrt{3}}{2}-\frac{\alpha_0}{2}}$$

the superposition then becomes

$$\implies
\left(\frac{\alpha_0\sqrt{3}}{2}-\frac{\alpha_1}{2}\right)\beta_0\ket{00}+\left(\frac{\alpha_1\sqrt{3}}{2}+\frac{\alpha_0}{2}\right)\beta_0\ket{10}+\left(\frac{\alpha_0\sqrt{3}}{2}-\frac{\alpha_1}{2}\right)\beta_1\ket{01}+\left(\frac{\alpha_1\sqrt{3}}{2}+\frac{\alpha_0}{2}\right)\beta_1\ket{11}$$

if we apply a constraint that
$\alpha_0\beta_0=\alpha_1\beta_1=\frac{1}{\sqrt{2}}$ and the 2 other
coefficients =0 we end up with our wave function as

$$\frac{\sqrt{6}}{4}\ket{00}+\frac{1}{2\sqrt{2}}\ket{01}+\frac{1}{2\sqrt{2}}+\frac{\sqrt{6}}{4}\ket{11}$$

Taking $\sum|\alpha_x|^2$ we get 3/4. So Alice and Bob win with .75
probability.\\

c) Show that if $x=y=1$, then Alice and Bob win the game with probability 3/4.\\

The superposition looks like
$$\frac{1}{2\sqrt{2}}\ket{00}+\frac{\sqrt{3}}{2\sqrt{2}}\ket{01}+\frac{\sqrt{3}}{2\sqrt{2}}\ket{10}+\frac{1}{2\sqrt{2}}\ket{11}$$

taking the sum of the squares of the coefficients we get a probability of .75\\

d) Combining parts a-c, conclude that Alice and Bob win with greater overall
probability that would be in a classical universe.\\

the weights of parts a-c (likelihood of happening) are; $x=y=0$ happens .25,
$x=1$ and $y=0$ or $x=0$ and $y=1$ happens .5, and $x=y=1$ happens .25. S0

$$.25(1)+.5(.75)+.25(.75)=.8125$$

This gives a probabilistic win percentage of \%81.25 which is higher than the
.75 we calculated in a classical universe.\\


\textbf{Problem 8:} Practice with three dimensional rotations\\

Recall (eq B.13) that the $2\times2$ unitary matrix $u(\hat
n,\theta)=\exp(i\frac{\theta}{2}\hat n,\vec\sigma)$ can be used to represent
rotations in three real dimensions by an angle $\theta$ about the axis
specified by the unit vector $\hat n$\\

a) Write down the matrices $u(\hat x,\pi)$, $u(\hat y,\pi)$ for rotations by
$180^\circ$ about the $\hat x$ and $\hat y$ axes. If you multiply these
matrices (and the one's they generate) in all possible ways, how many distinct
$2\times 2$ unitary matrices $u$ does this produce? To what three dimensional
rotations do these corresond?\\

from the book we have $u(\hat x,\pi)=\exp(i\frac{\pi}{2}\hat x\cdot\vec\sigma_x)$ and $u(\hat y,\psi)=\exp(i\frac{\pi}{2}\hat y\cdot\vec\sigma)y)$.

It is clear (a simple calculation would show) that these matricies do not
commute, therefore we can produce 4 distinct matrices from them. These
matrices correspond to a rotation about the $x$ axis by 180$^\circ$ and a
rotation about the $y$ axis by $180^\circ$. We can get to any other point on a
sphere via these transformations.\\

b) We have $u(\hat x,\pi)=\exp(i\frac{\pi}{4}\hat x\cdot\vec\sigma_x)$ and $u(\hat y,\psi)=\exp(i\frac{\pi}{4}\hat y\cdot\vec\sigma)y)$. Because these have more intervals before they hit a complete identity map. As a result we can enumerate 24 distict matricies up to isomorphism.\\













\end{document}
