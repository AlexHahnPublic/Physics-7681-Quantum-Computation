\documentclass[a4paper,11pt]{article}
\usepackage{amsmath}
\usepackage{amsthm}
\usepackage[parfill]{parskip} 
\usepackage{color}
\usepackage[norelsize,ruled, vlined]{algorithm2e}
\usepackage{amsfonts}
\usepackage{amssymb}
\usepackage{fancyhdr}
\usepackage{enumerate}
\usepackage{tikz}
\usepackage{pgfplots}
\usepackage{hyperref}
\usetikzlibrary{decorations.markings}
%    Q-circuit version 2
%    Copyright (C) 2004  Steve Flammia & Bryan Eastin
%    Last modified on: 9/16/2011
%
%    This program is free software; you can redistribute it and/or modify
%    it under the terms of the GNU General Public License as published by
%    the Free Software Foundation; either version 2 of the License, or
%    (at your option) any later version.
%
%    This program is distributed in the hope that it will be useful,
%    but WITHOUT ANY WARRANTY; without even the implied warranty of
%    MERCHANTABILITY or FITNESS FOR A PARTICULAR PURPOSE.  See the
%    GNU General Public License for more details.
%
%    You should have received a copy of the GNU General Public License
%    along with this program; if not, write to the Free Software
%    Foundation, Inc., 59 Temple Place, Suite 330, Boston, MA  02111-1307  USA

% Thanks to the Xy-pic guys, Kristoffer H Rose, Ross Moore, and Daniel Müllner,
% for their help in making Qcircuit work with Xy-pic version 3.8.
% Thanks also to Dave Clader, Andrew Childs, Rafael Possignolo, Tyson Williams,
% Sergio Boixo, Cris Moore, Jonas Anderson, and Stephan Mertens for helping us test
% and/or develop the new version.

\usepackage{xy}
\xyoption{matrix}
\xyoption{frame}
\xyoption{arrow}
\xyoption{arc}

\usepackage{ifpdf}
\ifpdf
\else
\PackageWarningNoLine{Qcircuit}{Qcircuit is loading in Postscript mode.  The Xy-pic options ps and dvips will be loaded.  If you wish to use other Postscript drivers for Xy-pic, you must modify the code in Qcircuit.tex}
%    The following options load the drivers most commonly required to
%    get proper Postscript output from Xy-pic.  Should these fail to work,
%    try replacing the following two lines with some of the other options
%    given in the Xy-pic reference manual.
\xyoption{ps}
\xyoption{dvips}
\fi

% The following resets Xy-pic matrix alignment to the pre-3.8 default, as
% required by Qcircuit.
\entrymodifiers={!C\entrybox}

\newcommand{\bra}[1]{{\left\langle{#1}\right\vert}}
\newcommand{\ket}[1]{{\left\vert{#1}\right\rangle}}
    % Defines Dirac notation. %7/5/07 added extra braces so that the commands will work in subscripts.
\newcommand{\qw}[1][-1]{\ar @{-} [0,#1]}
    % Defines a wire that connects horizontally.  By default it connects to the object on the left of the current object.
    % WARNING: Wire commands must appear after the gate in any given entry.
\newcommand{\qwx}[1][-1]{\ar @{-} [#1,0]}
    % Defines a wire that connects vertically.  By default it connects to the object above the current object.
    % WARNING: Wire commands must appear after the gate in any given entry.
\newcommand{\cw}[1][-1]{\ar @{=} [0,#1]}
    % Defines a classical wire that connects horizontally.  By default it connects to the object on the left of the current object.
    % WARNING: Wire commands must appear after the gate in any given entry.
\newcommand{\cwx}[1][-1]{\ar @{=} [#1,0]}
    % Defines a classical wire that connects vertically.  By default it connects to the object above the current object.
    % WARNING: Wire commands must appear after the gate in any given entry.
\newcommand{\gate}[1]{*+<.6em>{#1} \POS ="i","i"+UR;"i"+UL **\dir{-};"i"+DL **\dir{-};"i"+DR **\dir{-};"i"+UR **\dir{-},"i" \qw}
    % Boxes the argument, making a gate.
\newcommand{\meter}{*=<1.8em,1.4em>{\xy ="j","j"-<.778em,.322em>;{"j"+<.778em,-.322em> \ellipse ur,_{}},"j"-<0em,.4em>;p+<.5em,.9em> **\dir{-},"j"+<2.2em,2.2em>*{},"j"-<2.2em,2.2em>*{} \endxy} \POS ="i","i"+UR;"i"+UL **\dir{-};"i"+DL **\dir{-};"i"+DR **\dir{-};"i"+UR **\dir{-},"i" \qw}
    % Inserts a measurement meter.
    % In case you're wondering, the constants .778em and .322em specify
    % one quarter of a circle with radius 1.1em.
    % The points added at + and - <2.2em,2.2em> are there to strech the
    % canvas, ensuring that the size is unaffected by erratic spacing issues
    % with the arc.
\newcommand{\measure}[1]{*+[F-:<.9em>]{#1} \qw}
    % Inserts a measurement bubble with user defined text.
\newcommand{\measuretab}[1]{*{\xy*+<.6em>{#1}="e";"e"+UL;"e"+UR **\dir{-};"e"+DR **\dir{-};"e"+DL **\dir{-};"e"+LC-<.5em,0em> **\dir{-};"e"+UL **\dir{-} \endxy} \qw}
    % Inserts a measurement tab with user defined text.
\newcommand{\measureD}[1]{*{\xy*+=<0em,.1em>{#1}="e";"e"+UR+<0em,.25em>;"e"+UL+<-.5em,.25em> **\dir{-};"e"+DL+<-.5em,-.25em> **\dir{-};"e"+DR+<0em,-.25em> **\dir{-};{"e"+UR+<0em,.25em>\ellipse^{}};"e"+C:,+(0,1)*{} \endxy} \qw}
    % Inserts a D-shaped measurement gate with user defined text.
\newcommand{\multimeasure}[2]{*+<1em,.9em>{\hphantom{#2}} \qw \POS[0,0].[#1,0];p !C *{#2},p \drop\frm<.9em>{-}}
    % Draws a multiple qubit measurement bubble starting at the current position and spanning #1 additional gates below.
    % #2 gives the label for the gate.
    % You must use an argument of the same width as #2 in \ghost for the wires to connect properly on the lower lines.
\newcommand{\multimeasureD}[2]{*+<1em,.9em>{\hphantom{#2}} \POS [0,0]="i",[0,0].[#1,0]="e",!C *{#2},"e"+UR-<.8em,0em>;"e"+UL **\dir{-};"e"+DL **\dir{-};"e"+DR+<-.8em,0em> **\dir{-};{"e"+DR+<0em,.8em>\ellipse^{}};"e"+UR+<0em,-.8em> **\dir{-};{"e"+UR-<.8em,0em>\ellipse^{}},"i" \qw}
    % Draws a multiple qubit D-shaped measurement gate starting at the current position and spanning #1 additional gates below.
    % #2 gives the label for the gate.
    % You must use an argument of the same width as #2 in \ghost for the wires to connect properly on the lower lines.
\newcommand{\control}{*!<0em,.025em>-=-<.2em>{\bullet}}
    % Inserts an unconnected control.
\newcommand{\controlo}{*+<.01em>{\xy -<.095em>*\xycircle<.19em>{} \endxy}}
    % Inserts a unconnected control-on-0.
\newcommand{\ctrl}[1]{\control \qwx[#1] \qw}
    % Inserts a control and connects it to the object #1 wires below.
\newcommand{\ctrlo}[1]{\controlo \qwx[#1] \qw}
    % Inserts a control-on-0 and connects it to the object #1 wires below.
\newcommand{\targ}{*+<.02em,.02em>{\xy ="i","i"-<.39em,0em>;"i"+<.39em,0em> **\dir{-}, "i"-<0em,.39em>;"i"+<0em,.39em> **\dir{-},"i"*\xycircle<.4em>{} \endxy} \qw}
    % Inserts a CNOT target.
\newcommand{\qswap}{*=<0em>{\times} \qw}
    % Inserts half a swap gate.
    % Must be connected to the other swap with \qwx.
\newcommand{\multigate}[2]{*+<1em,.9em>{\hphantom{#2}} \POS [0,0]="i",[0,0].[#1,0]="e",!C *{#2},"e"+UR;"e"+UL **\dir{-};"e"+DL **\dir{-};"e"+DR **\dir{-};"e"+UR **\dir{-},"i" \qw}
    % Draws a multiple qubit gate starting at the current position and spanning #1 additional gates below.
    % #2 gives the label for the gate.
    % You must use an argument of the same width as #2 in \ghost for the wires to connect properly on the lower lines.
\newcommand{\ghost}[1]{*+<1em,.9em>{\hphantom{#1}} \qw}
    % Leaves space for \multigate on wires other than the one on which \multigate appears.  Without this command wires will cross your gate.
    % #1 should match the second argument in the corresponding \multigate.
\newcommand{\push}[1]{*{#1}}
    % Inserts #1, overriding the default that causes entries to have zero size.  This command takes the place of a gate.
    % Like a gate, it must precede any wire commands.
    % \push is useful for forcing columns apart.
    % NOTE: It might be useful to know that a gate is about 1.3 times the height of its contents.  I.e. \gate{M} is 1.3em tall.
    % WARNING: \push must appear before any wire commands and may not appear in an entry with a gate or label.
\newcommand{\gategroup}[6]{\POS"#1,#2"."#3,#2"."#1,#4"."#3,#4"!C*+<#5>\frm{#6}}
    % Constructs a box or bracket enclosing the square block spanning rows #1-#3 and columns=#2-#4.
    % The block is given a margin #5/2, so #5 should be a valid length.
    % #6 can take the following arguments -- or . or _\} or ^\} or \{ or \} or _) or ^) or ( or ) where the first two options yield dashed and
    % dotted boxes respectively, and the last eight options yield bottom, top, left, and right braces of the curly or normal variety.  See the Xy-pic reference manual for more options.
    % \gategroup can appear at the end of any gate entry, but it's good form to pick either the last entry or one of the corner gates.
    % BUG: \gategroup uses the four corner gates to determine the size of the bounding box.  Other gates may stick out of that box.  See \prop.

\newcommand{\rstick}[1]{*!L!<-.5em,0em>=<0em>{#1}}
    % Centers the left side of #1 in the cell.  Intended for lining up wire labels.  Note that non-gates have default size zero.
\newcommand{\lstick}[1]{*!R!<.5em,0em>=<0em>{#1}}
    % Centers the right side of #1 in the cell.  Intended for lining up wire labels.  Note that non-gates have default size zero.
\newcommand{\ustick}[1]{*!D!<0em,-.5em>=<0em>{#1}}
    % Centers the bottom of #1 in the cell.  Intended for lining up wire labels.  Note that non-gates have default size zero.
\newcommand{\dstick}[1]{*!U!<0em,.5em>=<0em>{#1}}
    % Centers the top of #1 in the cell.  Intended for lining up wire labels.  Note that non-gates have default size zero.
\newcommand{\Qcircuit}{\xymatrix @*=<0em>}
    % Defines \Qcircuit as an \xymatrix with entries of default size 0em.
\newcommand{\link}[2]{\ar @{-} [#1,#2]}
    % Draws a wire or connecting line to the element #1 rows down and #2 columns forward.
\newcommand{\pureghost}[1]{*+<1em,.9em>{\hphantom{#1}}}
    % Same as \ghost except it omits the wire leading to the left.

\DeclareMathAlphabet{\mathpzc}{OT1}{pzc}{m}{it}
\usepackage[margin = 1.0in]{geometry}
\pagestyle{fancy}
\lhead{Andrew Casey acc248}
\chead{CS 4812}
\rhead{HW2 2/25/14}

\newtheorem*{mydef}{Definition}
\newtheorem*{thm}{\\ Theorem}
\newtheorem*{lem}{\\ Lemma}
\newtheorem*{claim}{\\ Claim}
\newtheorem*{defn}{\\ Definition}
\newtheorem*{prop}{\\ Proposition}

\DeclareMathOperator{\N}{\mathbb{N}}
\DeclareMathOperator{\Z}{\mathbb{Z}}
\DeclareMathOperator{\Q}{\mathbb{Q}}
\DeclareMathOperator{\R}{\mathbb{R}}
\DeclareMathOperator{\F}{\mathbb{F}}
\DeclareMathOperator{\C}{\mathbb{C}}

\begin{document}
\subsubsection*{Problem 1}
\emph{\color{red} Come back and write out what the $|\Phi_x\rangle$ for one part just to demonstrate you know what it means and also reference (1.78) and (1.79)}
\begin{enumerate}[a)]
    \item The probability that the 3 Qbits are measured simultaneously to be (1,0,1) is given by $|\alpha_{101}|^2$. The Qbits will be left in state $|101\rangle$ after the measurement.
    \item The probability that Qbit 2 is measured to be (1) is given by $|\alpha_1|^2$ and the Qbits will be left in state $|1\rangle|\Phi_1\rangle_2$.
    \item The probability that Qbits 2,1 are measured simultaneously to be (1,0) is given by $|\alpha_{10}|^2$. The Qbits will be left in state $|10\rangle|\Phi_{10}\rangle_1$ after the measurement.
    \item The probability that Qbit 2 is measured to be (1) is given by $|\alpha_1|^2$ and the Qbits will be left in state $|1\rangle|\Phi_1\rangle_2$. Then, the probability that Qbit 1 is measure to be (0) is given by $|\alpha_0|^2$ and the Qbits will be left in state $|1\rangle|0\rangle|\Phi_{10}\rangle_1=|10\rangle|\Phi_{10}\rangle_1$.
    \item The probability that Qbits 2,1 are measured simultaneously to be (1,0) is given by $|\alpha_{10}|^2$. The Qbits will be left in state $|10\rangle|\Phi_{10}\rangle_1$ after the measurement. Then, the probability that Qbit 0 is measure to be (1) is given by $|\alpha_1|^2$ and the Qbits will be left in state $|1\rangle|0\rangle|1\rangle=|101\rangle$.
\end{enumerate}
\subsubsection*{Problem 2}
\begin{enumerate}[a)]
    \item We introduce $u$ such that $r+s+u=n$. We can then write:
        $$|\Psi\rangle_n=\sum\limits_{x,y,z}\alpha_{xyz}|x\rangle_r|y\rangle_s|z\rangle_u$$
        If we measure the first $r$ Qbits, we will measure $x$ with probability:
        $$p(x)=\sum\limits_{y,z}|\alpha_{xyz}|^2$$ 
        and the Qbits will be left in the state:
        $$|x\rangle_r|\Phi_x\rangle_{s+u}=|x\rangle_r\frac{1}{\sqrt{p(x)}}\sum\limits_{y,z}\alpha_{xyz}|y\rangle_s|z\rangle_u$$
        If we then measure the next $s$ Qbits, we will measure $y$ with probability:
        $$p(y|x)=\sum\limits_{z}|\frac{\alpha_{xyz}}{\sqrt{p(x)}}|^2$$ 
        and the Qbits will be left in the state:
        $$|x\rangle_r|y\rangle_s|\Phi_{xy}\rangle_{u}=|x\rangle_r|y\rangle_s\frac{1}{\sqrt{p(y|x)}}\frac{1}{\sqrt{p(x)}}\sum\limits_{z}\alpha_{xyz}|z\rangle_u$$
    \item If we measure $r+s$ Qbits all at once, we measure $xy$ with probability:
        $$p(xy)=\sum\limits_{z}|\alpha_{xyz}|^2$$
        and the Qbits will be left in the state:
        $$|x\rangle_r|y\rangle_s\frac{1}{\sqrt{p(xy)}}\sum\limits_{z}\alpha_{xyz}|z\rangle_u$$
        We then make the observation that $p(xy)=p(x)p(y|x)$ and thus $\frac{1}{\sqrt{p(y|x)}}\frac{1}{\sqrt{p(x)}}=\frac{1}{\sqrt{p(xy)}}$ observe that the result from part a is indeed the same as the result here.
    \item $$\alpha_x^2=\sum\limits_{0\leq x^\prime<2^{m+n}}|\gamma_{x^\prime}|^2$$
\emph{\color{red} Not sure about this one}
\end{enumerate}
\subsubsection*{Problem 3} 
\begin{enumerate}[a)]
    \item I would say that the setup of this question could provide some statistic sampling about the upper Qbit. After the first cNOT gate, the Qbits will become entangled and share the state:
        $$\alpha|00\rangle + \beta|11\rangle$$
        When we measure the lower Qbit, Born's rule tells us that we'll measure a 0 or 1 with probability $|\alpha|^2$ or $|\beta|^2$ respectively and subsequently unentangle the states and collapse the state of the lower Qbit. Now, the lower Qbit is in state $|0\rangle$ or $|1\rangle$. If it's in state $|0\rangle$. Then we fall into the same case as in observing the first gate. If it's in state $|1\rangle$ then we observe that the cNOT gate will act and the Qbits will become entangled and share the state:
        $$\alpha|01\rangle + \beta|11\rangle$$
        When we measure the lower Qbit now (which is represented here as Qbit 0), Born's rule tells us that we'll measure a 1 or 0 with probability $|\alpha|^2$ or $|\beta|^2$ respectively and subsequently unentangle the states and collapse the state of the lower Qbit. Thus, we see the patterns that can occur and observe that everytime we make a measurement, the lower Qbit will be the same as the last measurement with probability $|\alpha|^2$. It will be different from the last measurement with probability $|\beta|^2$. Thus, we simply keep track of the number of times that the measurement changes. If the measurement changes $x$ times, we can write our estimated probability $|\beta_e|^2$ as:
        $$|\beta_e|^2=\frac{x}{N}$$
        where our esimate has a margin of error ($\epsilon$) where $|\beta_e-\beta|<\epsilon$ at a confidence level denoted by $Z$ and given by the $Z$-value of a normal distribution. Note that $\epsilon=\frac{Z}{2\sqrt{N}}$, thus we can choose our confidence level and $N$ based on whatever error we would like to allow. This also allows us to find the value of $\alpha$ by the normalization constraint or by the same process as above (keeping track of the number of times the measurement doesn't change).
    \item This actually makes things worse as this adds a greater level of uncertainty to every measurement we're making.
\emph{\color{red} Do I need more than this?}
\end{enumerate}
\subsubsection*{Problem 4}
\begin{enumerate}[a)]
    \item If we utilize the identity $\mathsf{Z}R_\theta\mathsf{Z}=R_-\theta$, we observe: 
        $$F_\theta\equiv R_\theta\mathsf{Z}R_{-\theta}=R_\theta\mathsf{Z}\mathsf{Z}R_\theta\mathsf{Z}=R_\theta R_\theta\mathsf{Z}$$
        which in this form shows that we perform a reflection of a the vertical direction following by two rotations by $\theta$ which is exactly a relection about an axis at angle $\theta$ clockwise from the horizontal. We also observe that:
        $$F_\theta=\left( \begin{array}{cc} \cos^2\theta-\sin^2\theta & 2\sin\theta\cos\theta \\ 2\sin\theta\cos\theta & \sin^2\theta-\cos^2\theta \end{array} \right)$$
        Thus, $$F_{\pi/8}=\left( \begin{array}{cc} \frac{1}{\sqrt{2}} &  \frac{1}{\sqrt{2}}\\  \frac{1}{\sqrt{2}}& -\frac{1}{\sqrt{2}}\end{array} \right)=\mathsf{H}$$
        Next, if we place a a half wave plate $\mathsf{H}=F_{\pi/8}$ before a $P_v$ we observe that the fraction of initially $h$-polarized light that transmits is $\cos^2(\frac{\pi}{2}+\frac{2\pi}{8})=0.5$ and the fraction of $v$-polarized light that transmits is $\cos^2(\frac{2\pi}{8})=0.5$. Meanwhile the fraction of $\pi/4$-polarized light that transmits is equal to $\cos^2(\frac{\pi}{4}+\frac{2\pi}{8})=1$.
    \item \begin{enumerate}[(i)]
            \item An $h$-polarized photon emerges with a polarization angle of $\frac{\pi}{3}$. A $\pi/6$-polarized photon emerges with vertical polarization ($\frac{\pi}{2}$).
            \item If $n=1$ the photon will emerge vertically polarized with probability 1. Otherwise the photon will have no chance of emerging vertically polarized.
        \end{enumerate}
    \item \begin{enumerate}[(i)]
            \item The probability of passing through both $P_{\pi/4}$ and $P_v$ is $0.5*0.5=.25$.
            \item The probability of passing through all $N$ in the limit $N$ large is equal to 1. This can be seen as if we look at the first polarizer, for $N$ large, the angle of polarization goes to 0. The difference between each subsequent polarizer also goes to 0 and thus the probability of passing through each subsequent polarizer is 1.
\end{enumerate}
\end{enumerate}

\subsubsection*{Problem 5}
We start by illustrating the circuit diagram for the problem:
        $$\Qcircuit @C=1em @R=.7em {
\lstick{\ket{x}}& \multigate{2}{U_f} & \rstick{\ket{x}}\qw \\
\lstick{\ket{y}}& \ghost{U_f}& \rstick{\ket{y}}\qw \\
\lstick{\ket{z}}& \ghost{U_f} & \rstick{\ket{z\oplus f(\ket{xy})}}\qw
}$$
\begin{enumerate}[a)]
    \item 
        $\Qcircuit @C=1em @R=.7em {
& \multigate{2}{U_f} & \qw \\
& \ghost{U_f} & \quad\quad\quad=\qw \\
& \ghost{U_f} & \qw
}  \quad\quad\quad
        \Qcircuit @C=1em @R=1em {
             &\qw & \qw & \qw &\qw\\
             &\qw & \qw & \qw & \qw\quad\quad\quad\quad = f_{0000}\\
             &\qw & \qw & \qw & \qw
         }\quad\quad\quad\quad\quad\quad
         \Qcircuit @C=1em @R=.7em {
& \multigate{2}{U_f} & \qw \\
& \ghost{U_f} & \quad\quad\quad=\qw \\
& \ghost{U_f} & \qw
}  \quad\quad\quad
        \Qcircuit @C=1em @R=1em {
             &\qw & \ctrl{2} & \qw &\qw\\
             &\qw & \qw & \qw & \qw\quad\quad\quad\quad = f_{0101}\\
             &\qw & \targ & \qw & \qw
         }$
         \\\\
        $\Qcircuit @C=1em @R=.7em {
& \multigate{2}{U_f} & \qw \\
& \ghost{U_f} & \quad\quad\quad=\qw \\
& \ghost{U_f} & \qw
}  \quad\quad\quad
        \Qcircuit @C=1em @R=1em {
             &\qw & \qw & \qw &\qw\\
             &\qw & \ctrl{1} & \qw & \qw\quad\quad\quad\quad = f_{0011}\\
             &\qw & \targ & \qw & \qw
         }\quad\quad\quad\quad\quad\quad
         \Qcircuit @C=1em @R=.7em {
& \multigate{2}{U_f} & \qw \\
& \ghost{U_f} & \quad\quad\quad=\qw \\
& \ghost{U_f} & \qw
}  \quad\quad\quad
        \Qcircuit @C=1em @R=1em {
             &\qw & \ctrl{2} & \qw &\qw\\
             &\qw & \qw & \qw & \qw\quad\quad\quad\quad = f_{1010}\\
             &\targ & \targ & \qw & \qw
         }$
         \\\\
        $\Qcircuit @C=1em @R=.7em {
& \multigate{2}{U_f} & \qw \\
& \ghost{U_f} & \quad\quad\quad=\qw \\
& \ghost{U_f} & \qw
}  \quad\quad\quad
        \Qcircuit @C=1em @R=1em {
             &\qw & \qw & \qw &\qw\\
             &\qw & \ctrl{1} & \qw & \qw\quad\quad\quad\quad = f_{1100}\\
             &\targ & \targ & \qw & \qw
         }\quad\quad\quad\quad\quad\quad
         \Qcircuit @C=1em @R=.7em {
& \multigate{2}{U_f} & \qw \\
& \ghost{U_f} & \quad\quad\quad=\qw \\
& \ghost{U_f} & \qw
}  \quad\quad\quad
        \Qcircuit @C=1em @R=1em {
             &\qw & \qw & \qw &\qw\\
             &\qw & \qw & \qw & \qw\quad\quad\quad\quad = f_{1111}\\
             &\qw & \targ & \qw & \qw
         }$
    \item
\end{enumerate}
\subsubsection*{Problem 6}
\begin{enumerate}[a)]
    \item In the worst case, $2^{n-1}+1$ evaluations of $f$ are needed. This is because at worst, we would observe half of the evaluations as being the same and we would then have to observe one more evaluation to see whether $f$ is balanced or constant. In the best case, we would only need $2^1$ evaluations as we would simply check the first two evaluations and if they're different, we know $f$ is balanced. The probability of the worst case is given by:\\
        $$.50+.50\left(\frac{(\frac{n}{2}-1)!}{\frac{(n-1)!}{(\frac{n}{2}-1)!}}\right)=.50+.50\left(\frac{1}{(n-1)!}\right)$$
        The probability of the best case is given by:
        $$.50\left(\frac{\frac{n}{2}}{n-1}\right)=.50\left(\frac{n}{2n-2}\right)$$
    \item \emph{\color{red} Need to tex this}
\end{enumerate}
\subsubsection*{Problem 7}
For some preliminary observations, we say that Alice's Qbit can be described by $\alpha_0\ket{0}+\alpha_1\ket{1}$ and Bob's can be described by $\beta_0\ket{0}+\beta_1\ket{1}$. Their entangled state can thus be described by $\alpha_0\beta_0\ket{00}+\alpha_0\beta_1\ket{01}+\alpha_1\beta_0\ket{10}+\alpha_1\beta_1\ket{11}$. Then we can create the constraint \begin{align}
    \alpha_0\beta_0&=\frac{1}{\sqrt{2}}=\alpha_1\beta_1\\
    \alpha_0\beta_1&=0=\alpha_1\beta_0
\end{align}
\begin{enumerate}[a)]
    \item If $x=y=0$ then neither Alice nor Bob make any changes to their Qbits. Thus, $a\oplus b=0=xy$ with probability 1. This is because Alice and Bob share the state $\frac{1}{\sqrt{2}}(\ket{00}+\ket{11})$, and Alice/Bob can measure a 0 or a 1 with probability of $\frac{1}{2}$. However, once one of them measures a 0 or 1, the other one will then make the same measurement due to the entangled state. Thus, they will make the same measurement 100\% of the time.
    \item We show this part for $x=1$. We observe that Alice can apply her unitary matrix as follows:
        $$R_{\pi/6}\left(\begin{array}{c} \alpha_0 \\ \alpha_1\end{array}\right)=\left(\begin{array}{c} \frac{\alpha_0\sqrt{3}}{2}-\frac{\alpha_1}{2} \\ \frac{\alpha_1\sqrt{3}}{2}+\frac{\alpha_0}{2}\end{array}\right)$$
        Then, our entangled state becomes:
        $$\left(\frac{\alpha_0\sqrt{3}}{2}-\frac{\alpha_1}{2}\right)\beta_0\ket{00}+\left(\frac{\alpha_0\sqrt{3}}{2}-\frac{\alpha_1}{2}\right)\beta_1\ket{01}+\left(\frac{\alpha_1\sqrt{3}}{2}+\frac{\alpha_0}{2}\right)\beta_0\ket{10}+\left(\frac{\alpha_1\sqrt{3}}{2}+\frac{\alpha_0}{2}\right)\beta_1\ket{11}$$
        And if we apply our constraints from (1) and (2) we see that this simplifies to:
        $$\frac{\sqrt{6}}{4}\ket{00}+\frac{1}{2\sqrt{2}}\ket{01}+\frac{1}{2\sqrt{2}}\ket{10}+\frac{\sqrt{6}}{4}\ket{11}$$
        We then see that the probability that Alice and Bob make the same measurement is equal to $\frac{6}{16}+\frac{6}{16}=\frac{3}{4}=.75$. Then $a\oplus b = xy = 0$ and Alice and Bob win with probability .75.
    \item With both Alice and Bob applying their unitary matrices, and utilizing the constraints (1) and (2), we can see that the entangled state before measurement is:
        $$\frac{1}{2\sqrt{2}}\ket{00}+\frac{2\sqrt{3}}{4\sqrt{2}}\ket{01}+\frac{2\sqrt{3}}{4\sqrt{2}}\ket{10}+\frac{1}{2\sqrt{2}}\ket{11}$$
        We can then see that Alice and Bob will make different measurements with probability $\frac{3}{8}+\frac{3}{8}=\frac{3}{4}=.75$. Thus, $a\oplus b = 1=xy$ and Alice and Bob win $\frac{3}{4}$ of the time.
    \item We know that $x=y=0$ can occur with probability .25, $x\neq y$ with probability .50, and $x=y=1$ with probability .25. Thus, we can sum the likelihood they'll win for each event as follows:
        $$.25(1)+.50(.75)+.25(.75)=.8125$$
        Thus Alice and Bob win with overall probability of $.8125=\frac{13}{16}$.
    \item
\end{enumerate}
\subsubsection*{Problem 8}
\emph{\color{red} Started this one and now not so sure about it.}
\begin{enumerate}[a)]
\item $$\mathsf{u}(\hat{x},\pi)=e^{i\frac{\pi}{2}\bar{x}\cdot\vec{\sigma}}=\left(\begin{array}{cc} 1 & i \\ i & 1 \end{array}\right)\quad\quad\quad\mathsf{u}(\hat{y},\pi)=e^{i\frac{\pi}{2}\bar{y}\cdot\vec{\sigma}}=\left(\begin{array}{cc} 1 & e^{-i} \\ e^i & 1 \end{array}\right)$$
    $$\mathsf{u}(\hat{x},\pi)\cdot\mathsf{u}(\hat{y},\pi)=\left(\begin{array}{cc} 1+ie^i & i+e^{-i} \\ i+e^i & 1+ie^{-i} \end{array}\right)$$
    $$\mathsf{u}(\hat{y},\pi)\cdot\mathsf{u}(\hat{x},\pi)=\left(\begin{array}{cc} 1+ie^{-i} & i+e^{-i} \\ i+e^i & 1+ie^{i} \end{array}\right)$$
    \item
\end{enumerate}
\end{document}
