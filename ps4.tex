\input{/Users/AlexHahn/LatexReusableFiles/PhysicsHeader.tex}
\setlength{\parindent}{0pt}
\begin{document}
\title{Physics 4481-7681; CS 4912 Problem Set 4}
\author{Alex Hahn}
\date{\today}
\maketitle

\textbf{Problem 1:} Defeating RSA encryption with period finding. I\\
Here we examine RSA encryption and how it can be defeated by an efficient
period-finding program, by working everything out in a particular case. To get
a better feeling for what is involved, try doing this without a computer or
calculator.\\

(a) The two numbers Bob announces publicly are $N=55$ and $c=17$. Let
Alice's message $a$ be 9. What number between 1 and 54 is Alice's encrypted
message $b\equiv a^c(\text{mod }55)$? Rather than using a calculator, note that 17 in
binary is 10001, so you can work it out efficiently by listing $9^2$ mod 55,
$9^4$ mod 55 etc. Your write-up should list the values of all the powers of 9
modulo 55 that you had to calculate to construct $b$. (If easier, a few can be
given as negative numbers modulo 55.)\\

$$9\equiv9(\text{mod }55)$$
$$9^2\equiv81\equiv26(\text{mod }55)$$
$$9^4\equiv(9^2)^2\equiv26^2\equiv16(\text{mod }55)$$
$$9^8\equiv(9^4)^2\equiv16^2\equiv36(\text{mod }55)$$
$$9^{16}\equiv(9^8)^2\equiv36^2\equiv31(\text{mod }55)$$

Since we are looking for $9^{17}$ we can use modular arithmetic to show

$$9^{17}=9^1\cdot9^{16}\equiv9\cdot31\equiv279\equiv4(\text{mod }55)$$

so 4 is the encrypted message.\\

(b) Since you have in your head the computational resources needed to find the
prime factors of 44, you are in a position to find Bob's decoding number
$d$. What is it? (This can be done using the Euclidean algorithm described in
Appendix I.) Confirm that $b^d\equiv a(\text{mod }55)$, by the same process of
successively squaring used in (a).\\

Using Euclid's algorithm we have

$$40=17\cdot2+6$$
$$17=6\cdot2+5$$
$$6=5\cdot1+1 \rightarrow -2(\text{mod }40)$$
$$5=1\cdot5+0\rightarrow 5(\text{mod }40)$$
$$\rightarrow33(\text{mod }40)$$

so 33 is our answer for $d$ the decoding number. Now we check that
$4^{33}\equiv9(\text{mod }55)$

$$4\equiv4(\text{mod }55)$$
$$4^2\equiv16(\text{mod }55)$$
$$4^4\equiv(4^2)^2\equiv16^2\equiv36(\text{mod }55)$$
$$4^8\equiv(4^4)^2\equiv36^2\equiv31(\text{mod }55)$$
$$4^{16}\equiv(4^8)^2\equiv31^2\equiv26(\text{mod }55)$$
$$4^{32}\equiv(4^{16})^2\equiv26^2\equiv16(\text{mod }55)$$

from here we can see that
$$4^{33}=4^1\cdot4^{32}\equiv 4\cdot 16\equiv9\text{mod }55~\checkmark$$

(c) Eve, listening in to the public communications picks up Bob's publicly
announced $N=55$ and $c=17$ as well as Alice's encoded message $b$. Using her
quantum computer she calculates the period $r$ of $b$ modulo $N$. What is
$r$? Although you lack a quantum computer you can factor $N$ in your head, and
therefore know the order of $G_N$. Since $r$ must divide that order there are
not very many possibilities to examine.\\

$r$ must be one of the factors of 40 because $G_N=4\cdot10=40$. after
checking a few of them we find that

$$4^{10}\equiv1(\text{mod }55)$$
$$\therefore ~r=10$$

(d) Find the inverse $d'$ of $c$ modulo $r$. (This is simple enough not to
need Euclid's algorithm.) Confirm that $b^{d'}\equiv a(\text{mod }55)$

$$17d'\equiv1(\text{mod }10)$$
$$7d'\equiv1(\text{mod }10)$$
$$7d'\equiv21(\text{mod }10)$$
$$d'\equiv3(\text{mod }10)$$

confirming $b^{d'}\equiv a(\text{mod }55)$:

$$4^3\equiv64(\text{mod }55)$$
$$4^3\equiv9(\text{mod }55)~\checkmark$$

\textbf{Problem 2:} Defeating RSA encryption with period finding. II\\
Now suppose Bob sends $N=143$ and $x=53$ over a public channel to Alice, who
uses them to encode her message $a$ and sends back the encoded message
$b=19$ to Bob. (In the below, feel free to use a classical computer or
classical pocket calculator if useful.)\\

(a) Eve uses her quantum computer to determine that the period of the
function $f(x)\equiv19^x(\text{mod }143)$ is $r=60$. Find $d'$ (using, e.g.,
Euclid's algorithm) such that $cd'\equiv1(\text{ mod}r)$, and use that to
recover Alice's original message $a$.\\

Using Euclid's algorithm:

$$60=53\cdot1+7$$
$$53=7\cdot7+4$$
$$7=4\cdot1+3\rightarrow-1\text{mod }40$$
$$4=3\cdot1+1\rightarrow8\text{mod }40$$
$$3=1\cdot3+0\rightarrow-9\text{mod }40$$
$$\rightarrow17\text{mod }40$$

therefore $d'=17$ and recovering Alice's original message we have
$19^{17}\equiv2(\text{mod }143)$ so $a=2$

(b) Bob knows instead the prime factors $p$, $q$ of $N$. Use those together
with Euclid's algorithm to determine $d$ such that $cd\equiv1(\text{mod
}(p-1)(q-1))$, then use $d$ to decrypt Alice's original message $a$, and see if
Eve got it right.\\

Using Euclid's algorithm

$$120=53\cdot2+14$$
$$53=14\cdot3+11$$
$$14=11\cdot1+4\rightarrow-2$$
$$11=3\cdot3+2\rightarrow7$$
$$3=2\cdot1+1\rightarrow-9$$
$$2=1\cdot2+0\rightarrow34$$
$$\rightarrow-43$$

so $d$, the decrypting code is -43. Checking $19^{-43}\equiv2(\text{mod
}143)$ so Eve was correct.\\

\textbf{Problem 3:} Discrete Fourier Transform\\
In class (lecture 14, and noted linked from course website,) we emulated the
steps to factor the number $N=15$ via period finding. Here we will consider two
problems related to factoring $N=21$:\\

(a) First we'll consider the function $f(x)=b^x$mod 21, with $b=4$. For
experience with discrete Fourier transform, imagine starting from a state
$\ket{\Phi}=\frac{1}{\sqrt{18}}\sum^{17}_{x=0}\ket{x}\ket{f(x)}$ (though
note this state can't be easily arranged with Hadamard gates in a
conventional Qbit quantum computer, since 18 is not a power of 2).\\

(i) Suppose we measure the ``output" $\ket{f(x)}$ as 16. In what state
$\ket{\Psi}$ will that leave the ``input"? Writing the state in the form
$\ket{\Psi}=\sum^{17}_{x=0}\gamma(x)\ket{x}$, graph the (discrete) function
$\gamma(x)$ for values from 0 to 17.\\

The input is left in the state

$$\ket{\Psi}=\frac{1}{\sqrt{6}}(\ket{2}+\ket{5}+\ket{8}+\ket{11}+\ket{14}+\ket{17})$$

We can visualize this in the frequency domain as a series of delta functions
with amplitude $\frac{1}{\sqrt{6}}\approx4.0824$ at the corresponding states

\begin{center}
\includegraphics[scale=.12]{/Users/AlexHahn/Desktop/graph1.JPG}
\end{center}


(ii) Now calculate the discrete Fourier transform
$\tilde\gamma(y)=\frac{1}{\sqrt{18}}\sum^{17}_{x=0}e^{2\pi i
xy/18}\gamma(x)$ (as given in eq 3.20 of course text), and graph the values of
the modulus squared $|\tilde\gamma(y)|^2$ for $y$ from 0 to 17.\\

writing a program in Matlab to calculate the sum over the terms I get an
answer of\\
\texttt{1.93e-15(Re)+8.37e-16(Im)} with the
corresponding graph:

\begin{center}
\includegraphics[scale=.12]{/Users/AlexHahn/Desktop/graph2.JPG}
\end{center}

(iii) Imagine now measuring the state
$U_{FT}\ket{\Psi}=\sum^{17}_{x=0}\tilde\gamma(y)\ket{y}$. With what
probability would the various non-zero values of $y$ be measured?\\

The non-zero values of $y$ each have probability $\frac{1}{3}$.\\

(iv) Unlike a classical discrete Fourier transform, in the quantum case the
measurement in part (iii) only reveals a single Fourier transform component.
Presuming you didn't already know the value of the period $r$ of the above $f(x)$, how could you infer the period from any one of those possible measured values of $y$?\\

(section 3.7)\\

So we run into a question of finding the period $r$ in a one computation
"take all" scheme. The quantum Fourier transform provides us with a unitary
transformation that takes the $x_0$ dependence into a ``harmless" overall
phase factor. Now to find $r$ we note that the probability function
$$p(y)=\frac{1}{2^nm}\left|\sum^{m-1}_{k=0}e^{2\pi ikry/2^n}\right|^2$$

is a function of the integer $y$ whose magnitude has maxima when $y$ is
close to integral multiples of $2^n/r$. Assuming that we have taken enough
input registers ie $2^n$ is as large as $N^2$ so it is large enough to
contain at least $N$ full periods we now look at equation 3.57

$$\left|\frac{y}{2^n}-\frac{j}{r}\right|\leq\frac{1}{2^{n+1}}$$

Here we know $n$ and $y$ is the result of our measurement so we have an
estimate for $j/r$. BUT because we have used twice as many Qbits as needed to
represent all the integers up to $N$ we have ensured that our estimate $j/r$ is
off by no more than $1/(2N^2)$. From here we can figure out $j/r$ (with our
knowledge of $y(measured)/18$) with the
method of continued fractions (detailed in appendix K). Note that the method
gives us $j_0$ and $r_0$ with no common factors such that $j_0/r_0=j/r$. We
then use the fact (result from Appendix J) that two random numbers $j$ and $r$
have a fairly decent chance of having no common factors. We check to see
whether $r_0$ is infact $r$ by seeing if $b^{r_0}$(mod $N$)$=b$. If this is not
the case we can smaller low multiples (it is rare that two random numbers
share large factors). If even this doesn't work out that great the book
details how we can repeat the measurement procedure and check the math to see
if we found $r$ and in fact if we have enough additional Qbits we have a
very high chance of finding $r$.\\

(b) (i) there are 10 such values, $\ket{\Psi}$ (the input state) looks like

$$\ket{\Psi}=\frac{1}{\sqrt{10}}(\ket{4}+\ket{10}+\ket{16}+\ket{22}+\ket{28}+\ket{34}+\ket{40}+\ket{46}+\ket{52}+\ket{58})$$


(ii) Note first off that $m$ is 10 and $n=5$ for us in the equation (from
book)

$$p(y)=\frac{1}{2^nm}\frac{\sin^2(\pi\delta mr/2^n)}{\sin^2(\pi\delta r/2^n)}$$

plugging in our values we indeed get

$$p(y)=\frac{1}{2^5\cdot10}\frac{\sin^2(\pi6\cdot10 y/2^5)}{\sin^2(\pi
6y/2^5)}$$
$$=\frac{1}{640}\frac{\sin^2(\pi60 y/64)}{\sin^2(\pi6y/64)}$$

Using MATLABS handy anonymous function capability
\begin{verbatim}
f=@(y)(1/640)*(((sin((pi*60*y)/64))^2)/((sin((pi*6*y)/64))^2))
\end{verbatim}

we can quickly find the 5 most likely $y$ value probabilities (just iterate
through 0 to 64, storing in an array, and sort)\\

\begin{verbatim}
arr=[];
for i= 1:63 %y=0 returns NaN
    arr(i)=f(i);
end

sarr=sort(arr);

\end{verbatim}




I find that the top 5 probabilities are all .1124 (I actually get that one
value is 1.3451? also the sum of all entries is 2.0326, which is $>1$, are we supposed to
normalize?). If we take the top 5 (including the 1.3451) we get a sum of
1.7947 (this is pretty high out of a total of 2 so it's a good chance we
return one of these) (without it we get about .6)\\

(iii) We follow the same logic/ procedure as explained in 3(a iv) but now
with $y/64$. or you could try to reverse solve

$$p(y_i)=\frac{1}{r}\left(\frac{\sin(\pi\delta_j)}{\pi\delta_j}\right)^2$$

(since we have the probabilities ?)\\


\textbf{Problem 4:} Factoring a product $(2^k+1)(2^l+1)$\\

Suppose we are asked to factor the number $N=16843009$, which is somehow
known in advance to be the product of two primes of the form $p=2^k+1$, and
$q=2^l+1$. This is the special case mentioned on pp. 81-82 of the course
text, in which the period is necessarily a power of 2 (since it divides
$(p-1)(q-1)$). The smallest $n_0$ such that $2^{n_0}>N$ in this case is
$n_0=25$, and as explained in the texr we can take the number of Qbits to be
$n=n_0$ (rather than $2n_0$). Feel free to uses a classical computing aid for
the arithmetic below, and although $N$ is small enough to factor directly by
classical methods, try instead to follow the steps of the quantum argument.\\

(a) Following the method in section 3.10, suppose we pick $a=255$ (coprime to
$N$), and we wish to find the period $r$ of the function $a^x$mod $N$ using our
$n=25$ bit quantum Fourier transformer. Find the total probability of
measuring $y$ to be an integer multiple of $2^n/r$, verify the argument on p 81
that this probability is ``essentially" unity, With what probability does
measuring $y$ give no useful information to determine $r$?\\

From the book we have
$$p(y_j)=\frac{1}{2^nm}\left(\frac{\sin(\pi\delta_j)}{\pi\delta_jr/2^n}\right)^2=\frac{1}{r}\left(\frac{\sin(\pi\delta_j)}{\pi\delta_j}\right)^2$$

We also know that if $y_j$ is an multiple of $2^n/r$ then $\delta_j=0$ and
that there are $r-1$ terms in this summation. Lastly we will use the fact
that $r$ is large. Our summation boils down to:

$$\sum_{j=1}^{r-1}p(y_j)=\frac{1}{2^nm}\left(\frac{\sin(\pi\delta_j)}{\pi\delta_jr/2^n}\right)^2=\frac{1}{r}\left(\frac{\sin(\pi\delta_j)}{\pi\delta_j}\right)^2=\frac{1}{r}\approx1~\checkmark$$

(b) Now suppose we measure $y=7\cdot2^19$. This determines $j/r$ directly
without having to use continued fractions, and gives $r$ up to any common
divisors with $j$. Verify that the $r$ you obtained does indeed satisfy
$a^r$ mod $N\equiv1$.\\

rearranging we have

$$r=\frac{7\cdot2^{25}}{2^{19}}$$
$$=448$$

checking in wolfram alpha $255^{448}$ mod 16843009 does equal 1 $\checkmark$\\

(c) Evaluate $t=a^{r/2}$ mod $N$, and use it to determine the prime factors of
$N$ by the method in section $3.10$, by finding the greatest common divisors of
$N$ with $t+1$ and $t-1$ ($a=255$ has already been chosen so that
$t+1\not=0$mod $N$, as specified in eq (3.67)).\\

$$t=a^{r/2}\text{mod } N=65536=2^{16}$$
$$p=(N,t-1)=257$$
$$q=(N,t+1)=2^{16}+1=65537$$


\textbf{Problem 5:} Period Finding and continued fractions\\

This illustrates the mathematics of the final (post-quantum-computational)
stage of Shor's period finding procedure, as described on p.82 and appendix K
of the course text. Make use of the theorem, cited in the text, that if $j$ and
$r$ have no common factors, and $x$ is an estimate for the fraction $j/r$
that differs from it by less than $1/2r^2$, then $j/r$ will appear as one of the
partial sums in the continues fraction expansion of $x$\\

(a) Suppose you know that the integer $r$ is less than 100 and that 7080 is
within $\frac{1}{2}$ of an integral multiple of $2^{14}/r$. What is $r$?\\

$x=\frac{7080}{2^{14}}$

$$x=\cfrac{1}{2+\cfrac{1}{3+\cfrac{1}{5+\cfrac{1}{2+\cfrac{1}{...}}}}}$$

the third iter has a partial sum of $35/81$, the iteration after that was a
denominator that is too big. Therefore we've found that since 81 is the only
multiple of 81 which is also less than 100. A quick check with wolfram alpha
does check out that $(35/81)(2^{14}$ is within half of 7080\\



(b) Suppose you know that the integer $r$ is less than 100 and that 2979 and
14564 are both within $\frac{1}{2}$ of integral multiples of $2^{14}/r$.
What is $r$?\\

$x=\frac{2979}{2^{14}}$\\

$$x=\cfrac{1}{5+\cfrac{1}{2+\cfrac{1}{1489...}}}$$
and\\

$z=\frac{14564}{2^{14}}$

$$x=\cfrac{1}{1+\cfrac{1}{8+\cfrac{1}{455...}}}$$

for both $x$ and $z$ we can see that the partial sum of $a_2$ is too big. For
$x$ we get $a_1=\frac{2}{11}$ which checks out in wolfram alpha for
$\frac{2}{11}(2^{14})$ to be within
half of 2979. For $z$ we get $a_1=\frac{8}{9}$ which also checks out that
$\frac{8}{9}(2^{14}$ is within half of 14564. From here we can say that $r$
must be 99 since $[9,11]=99$ (and is infact less than 100)\\


\textbf{Problem 6:} Number of gates\\
(a) Fig. 3.1 shows the gates required to implement $U_{FT}$ fro $n=4$ Qbits. How
many Hadamard operations $H$ are required, and how many phase gates $V$ are
required? What is the total number of gates? For $n=5$, how many $H$'s and
$V$'s, and total gates would be required? For arbitrary $n$, how many would be
required? (In each of the above, total gates just means $H$'s plus $V$'s,
with each $V$ considered a single gate.)\\

With $n=4$ we need 4 $H$ gates and 6 $V$ gates giving us a total of 10 gates in
all. With $n=5$ we need 5 $H$ gates and 10 $V$ gates giving us a total of 15
gates in all. For an arbitrary $n$ we need $n$ $H$ gates and
$\frac{n(n-1)}{2}$ $V$ gates so we need

$$n+\frac{n(n-1)}{2}$$
$$=\frac{2n+n^2-n}{2}$$
$$=\frac{n^2+n}{2}~\text{gates}$$

(b) After eq 3.63, it was argued that eliminating all $V$ gates connecting
Qbits more than $l=22$ wires apart permitted factoring a 500 decimal digit
number with at most a 1\% loss in the probability of learning the period. How
large should $l$ be to permit factoring the largest of the RSA numbers, with 617
decimal digits (i.e., of order 100 quadrillion (googol)$^6$) with no more
than a 1\% loss in the probability? How many gates would be required (for the
quantum Fourier transform part) for factoring the 617 decimal digit $RSA$
number, eliminating all gates above the threshold value of $l$?\\

using the model example of $1/2^l<1(500n\pi)$,  we might
need $n$ to be around 4000 (the text references 3000 for a 500-digit $N$) for
a 617 digit $N$. Plugging  these numbers wofram spits put an $l$ slightly
larger than 22.\\

(c) For fun, estimate how long it would take to factor such a number using
conventional techniques. The largest RSA number factored has 232 decimal
digits, and was factored in late 2009, as described here (link). This took on
the order of $10^{20}$ operations in roughly two years, hence corresponded to
the equivalent of using roughly 1000, 2.2GHz single core processors over
that time-frame. Assuming that the number of operations to factor a number
with $d$ decimal digits scales as $C$exp($\beta d^{1/3}$) (where $C$ and
$\beta$ are some constants), and using the estimate in the above reference
that a 309 decimal digit number would take about 1000 times longer, how long
would it take to factor a number with 617 decimal digits using the same
technology?\\

plugging in the system

$$Ce^{\beta232^{1/3}}=2\text{ years}$$
$$Ce^{\beta309^{1/3}}=2000\text{ years}$$

we get $C=2.4\times 10^{-30}$ and $\beta=11.21$. Now calculating

$$2.4\times10^{-30}e^{11.21\cdot617^{1/3}}$$

we get that it would take 6.71$\times10^{11}$ years to finish. 6.7 hundred
billion years is a pretty long time...\\

\textbf{Problem 7:} In sec 3.8 of the text (p.83-84), it's shown how to
efficiently calculate the periodic function $f(x)=b^x$ mod $N$, highlighting
that it can be done with a single execution of the procedure that does the
multiple squarings (without need of a classical look-up table).\\

(i) Draw a circuit diagram for how this would be done for $N=15$, in terms of
black box quantum circuits that implement controlled multiplication (step a),
and mod $N$ square (step b).\\

\begin{center}
\includegraphics[scale=.12]{/Users/AlexHahn/Desktop/circuit.JPG}
\end{center}

(ii) As mentioned in class, the modular exponentiation to calculate the
periodic function itself might be more gate-intensice than the Fourier
transform (to calculate its period). Estimate how the number of operations
scales in the number of digits of the number to be factored, and roughly how
many total gates (modular exponentiation plus quantum Fourier transform)
might be necessary to factor the largest of the RSA numbers (as considered in
problem 6b above).\\
Note: the details will depend on the implementation, this is just to
estimate some rough number\\

After doing a little research into the matter it looks like it should be
polynomial time. (ie the time taken is polynomial in log $N$, which is the
size of the input). A quick search puts it at $O$((log $N$)$^3$). The
efficiency is mainly due to the QFT and modular exponentiation by repeated
squarings.\\

Now for an implied complication of this problem it is true that using
repeated squares for modular exponentiation requires a fair number of Qbits and
many more gates than the Fourier transform. It looks like the circuit
implementation for the QTF involves $q(q-1)/2$ gates.\\\\\\\\




*(Talked about a few of the problems with Andrew Casey (and a couple other
kids, but mainly Andrew))






\end{document}
