\input{/Users/AlexHahn/LatexReusableFiles/PhysicsHeader.tex}
\begin{document}
\title{Quantum Information Processing: Problem Set 6}
\author{Alex Hahn}
\date{\today}
\maketitle
\setlength{\parindent}{0pt}

\textbf{Problem 1}: Bell state via $\bf{X}_1\bf{X}_0$ and
$\bf{Z}_1\bf{Z}_0$ measurement\\

Consider a system of 2-Qbits. Show that the operators
$\bf{X}_1\bf{X}_0$ and $\bf{Z}_1\bf{Z}_0$ commute and that they square to
$\bf{1}$. We can therefore ``measure the two operators in the usual way,
projecting onto their joint $\pm1$ eigenspaces:

\begin{center}
\includegraphics[scale=.6]{/Users/AlexHahn/Desktop/1.png}
\end{center}

The left-hand-side circuit in the above figure is the usual
prescription for ``measuring" operators whose square is $\bf{1}$ (eqns
5.21-5.23 of text). Show that the circuit on the right-hand-sid eis
equivalent in this case.\\

Starting from an arbitrary state
$\ket{\Psi}=\alpha\ket{00}+\beta\ket{01}+\delta\ket{10}+\gamma\ket{11}$,
show that the two measurements specify the same four Bell states as given in
eqs. 6.3-6.5 of the text. To
do this, write down the full 4-Qbit initial state, then carefully work through
the circuit (in either of the above forms) to determine the final state, and show what
happens for each of the four possible $x,y$ measurements.\\
(Note that this is a simplified version of the four Qbit measurements
$\bf{X}_a\bf{X}_b\bf{X}_c\bf{X}_d$ and $\bf{Z}_a'\bf{Z}_b'\bf{Z}_c'\bf{Z}_d'$ to be discussed in class in the description of the surface code, and illustrates the same essential features.)\\

First we note that

$$\text{commutivity relation}=\left\{\begin{array}{lr}
(\bf{X}_i,\bf{Z_j})=1 & \text{, when } i\not=j\\
(\bf{X}_i,\bf{Z_j})=-1 & \text{, when } i=j
\end{array}\right.$$

From here we can form the relation

$$\bf{X}_1\bf{X}_0\bf{Z}_1\bf{Z}_0$$
$$=-\bf{X}_1\bf{Z}_1\bf{Z}_0\bf{X}_0$$
$$=\bf{Z}_1\bf{Z}_0\bf{X}_1\bf{X}_0$$

since $\bf{X}$'s and $\bf{Z}$'s commute with themselves respectively we have
that
$$\bf{X}_1\bf{X}_0\bf{X}_1\bf{X}_0$$
$$=\bf{Z}_1\bf{Z}_0\bf{Z}_1\bf{Z}_0$$

applying $\bf{X}_1\bf{X}_0$ to the state we have

$$\frac{1}{2}(\ket{00}\ket{\Psi}+\ket{00}\ket{\Psi}+\ket{10}\ket{\Psi}-\ket{10}\ket{\Psi})$$

As for the RHS of the circuit

$$\bf{C}_{02}(\bf{C}_{12}\otimes \bf{H}_3)\bf{C}_{30}\bf{C}_{31}\bf{H}_3 \ket{00}\ket{\Psi}$$
$$=\bf{C}_{02}(C_{12}\otimes
H_3)C_{30}C_{31}\frac{1}{\sqrt{2}}(\ket{00}\ket{\Psi}+\ket{10}\ket{\Psi})$$
$$=\bf{C}_{02}(C_{12}\otimes
H_3)\frac{1}{\sqrt{2}}(\ket{00}\ket{\Psi}+\ket{10}\ket{\Psi})$$
$$=\frac{1}{2}C_{02}(\ket{00}\ket{\Psi}+\ket{01}\ket{\Psi}+\ket{10}\ket{\Psi}-\ket{11}\ket{\Psi})$$
$$=\frac{1}{2}(\ket{00}\ket{\Psi}+\ket{00}\ket{\Psi}+\ket{10}\ket{Psi}-\ket{10}\ket{\Psi})$$

Which does equal the LHS. Writing out $\ket{\Psi}$ as a superposition we can
observe the final state for the $x,y$ measurements:
$$\frac{1}{\sqrt{2}}(\alpha\ket{00}+\beta\ket{01}+\delta\ket{10}+\gamma\ket{11})+\frac{1}{\sqrt{2}}(\gamma\ket{00}+\delta\ket{01}+\beta\ket{10}+\alpha\ket{11})$$

\textbf{Problem 2}: Double flip error correction\\

Suppose the only kinds of errors one had to worry about were bit-
ip errors, but one
wanted to take into account not only single bit-flips ($\textbf{X}_i$), in the
code words but also double bit-flips ($\textbf{X}_i\textbf{X}_j$,
$i\not=j$).\\

(a) Show that there is an $n$ for which the dimension of the $n$-Qbit state space is
just large enough to accommodate mutually orthogonal two-dimensional
subspaces for
the uncorrupted code words and all code words suffering either single or double
bit-flip
corruptions. What is that $n$?\\

A couple classmates have pointed out that this is analogous to the
formulation of 5.28 in the text. We are taking a codeword and putting it in a
1+3$n$ orthogonal 2d subspace, one 2d subspace for unperturbed
$\ket{\Psi}$ and $3n$ subspaces for each 1 and 2 Qbit flips. We argue from
here that the $2^k$d space spanned by the states of the $n$ Qbits
$\subseteq$  $1+3n$ 2d subspaces implying $2^{n-1}\geq3n+1~\implies~n=5$.\\


(b) Show for the $n$ you found in (a) that there is indeed a ``perfect" $n$-Qbit
code that
corrects all single and double bit-
ip errors, by writing down the states that encode $\ket{\bar{0}}$ and
$\ket{\bar{1}}$, and writing down a set of commuting hermitian operators whose squares
are unity, that
preserve both codewords, and have distinct patterns of commutations or
anticommutations
(show this explicitly) with each of the operators associated with all the
single and double
bit-flip errors.\\

For codewords $\ket{00000}$ and $\ket{11111}$ and hermitian operators
$\textbf{M}_{0-4}$ created by the set of cycling of
$\{\textbf{Z}_0,\textbf{Z}_1,\textbf{Z}_2,\textbf{Z}_3,\textbf{Z}_4\}$ we get
the commutivity table:

$$\begin{array}{cccccc}
 & \textbf{Z}_0\textbf{Z}_1\textbf{Z}_2\textbf{Z}_3 &
 \textbf{Z}_1\textbf{Z}_2\textbf{Z}_3\textbf{Z}_4 &
 \textbf{Z}_2\textbf{Z}_3\textbf{Z}_4\textbf{Z}_0 &
 \textbf{Z}_3\textbf{Z}_4\textbf{Z}_0\textbf{Z}_1 &
 \textbf{Z}_4\textbf{Z}_0\textbf{Z}_1\textbf{Z}_2\\
 \textbf{X}_0 & - & + & - & - & -\\
 \textbf{X}_1 & - & - & + & - & -\\
 \textbf{X}_2 & - & - & - & + & -\\
 \textbf{X}_3 & - & - & - & - & +\\
 \textbf{X}_4 & + & - & - & - & -\\
 \textbf{X}_0\textbf{X}_1 & + & - & - & + & +\\
 \textbf{X}_0\textbf{X}_2 & + & - & + & - & +\\
 \textbf{X}_0\textbf{X}_3 & + & - & + & + & -\\
 \textbf{X}_0\textbf{X}_4 & + & - & + & + & +\\
 \textbf{X}_1\textbf{X}_2 & - & + & - & - & +\\
 \textbf{X}_1\textbf{X}_3 & + & + & - & + & -\\
 \textbf{X}_1\textbf{X}_4 & - & + & - & + & +\\
 \textbf{X}_2\textbf{X}_3 & + & + & + & - & -\\
 \textbf{X}_2\textbf{X}_4 & - & + & + & - & +\\
 \textbf{X}_3\textbf{X}_4 & - & + & + & + & -\\
 \end{array}$$


 \textbf{Problem 3}: Simple exercises in use of tables 5.2 and 5.3\\
 (a) Suppose you are using a 5-Qbit error correcting code, and the
 operators ($\textbf{M}_0,~\textbf{M}_1,~\textbf{M}_2,~\textbf{M}_3$) are
 measured to have the following eigenvalues. In each case, what operator
 should be applied to correct the state?\\

(i) (+,-,+,-) = $\textbf{X}_2$\\
(ii) (-,-,+,+) = $\textbf{Z}_3$\\
(iii) (+,-,-,-) = $\textbf{Y}_0$\\

(b) In each of these cases of preparing states in the 5-Qbit code, what
eigenvalues of
the 4 $\textbf{M}_i$ and $\bar{\textbf{Z}}$ were measured?\\
(i) In order to prepare the state $\ket{\bar{1}}$, you determine that you need
to apply $\textbf{X}_3$.\\

(+,+,-,+),-\\

(ii) To prepare the state $\ket{\bar{0}}$, you determine that you need to apply
$\textbf{Y}_2$.\\

(-,-,+,-),+\\


(c) ...In each case, what operator should be applied to correct the state?\\

(i) (+,+,+,+,+,-) = (+,+,+,-,-,+)\\

(ii) (+,-,-,+,+,+) = (-,+,+,-,+,+)\\

(iii) (-,+,-,-,+,-) = (-,-,-,+,+,+)\\

\textbf{Problem 4}: Logical Operations on Codewords\\
(consulted Andrew Casey on this problem)\\

(a) ... Show that $\bar{\textbf{H}}=\textbf{H}^{\otimes5}$ does not
implement the logical Hadamard on the codewords in the case of the 5-Qbit
code.\\

In the book it makes clear that the presence of $\textbf{X}$ and
$\textbf{Z}$ make this 5-Qbit analog impossible. To be precise to get  the
$\bar{\textbf{X}}$ through the operators, the $\textbf{Z}_0$ and
$\textbf{Z}_3$ (anti-com) won't cancel if the 0 and 3 bit are
different. (observe parity of $\textbf{Z}$'s)\\


(b)

\begin{center}
\includegraphics[scale=.07]{/Users/AlexHahn/Desktop/2.JPG}
\end{center}

To show that the states are left invariant by $\textbf{M}_1$ observe

$$\ket{\bar{0}}=2^{-3/2}(\textbf{M}_1\textbf{M}_0+\textbf{M}_1\textbf{M}_2\textbf{M}_0+\textbf{M}_2\textbf{M}_0+\textbf{M}_0+\textbf{M}_1\textbf{M}_1\textbf{M}_2+\textbf{M}_2+1)\ket{0}_7$$

and

$$\ket{\bar{1}}=2^{-3/2}(\textbf{M}_1\textbf{M}_0+\textbf{M}_1\textbf{M}_2\textbf{M}_0+\textbf{M}_2\textbf{M}_0+\textbf{M}_0+\textbf{M}_1+\textbf{M}_1\textbf{M}_2+\textbf{M}_2+1)\bar{\textbf{X}}\ket{\bar{0}}$$

it is also shown that $\ket{\bar{1}}=\bar{\textbf{X}}\ket{\bar{0}}$

\textbf{Problem 5}: Experimental realization of W-state
$\ket{W}=\frac{1}{\sqrt{3}}(\ket{001}+\ket{010}+\ket{100})$\\

(a) Write down a circuit to construct the 3-qubit W-state $\ket{W}$.

\begin{center}
\includegraphics[scale=.08]{/Users/AlexHahn/Desktop/3.JPG}
\end{center}

5 (b) Another way to construct the W-state is give by

\begin{center}
\includegraphics[scale=.8]{/Users/AlexHahn/Desktop/4.png}
\end{center}

Show that the above circuit produces the W-state up to an overall
unobservable phase.\\


- Struggled with a classmate for a while on it, don't really have a great
answer\\

\textbf{Problem 6}: Teleportation ``through" a Gate\\

(a) Draw a circuit diagram of the six Qbits to show that if Bob applies the
correct operators, then he'll end up with the state
$\textbf{S}\ket{\psi_1}\ket{\psi_0}$ (i.e. the states are teleported
``through" the gate $\textbf{S}$). In particular show that if
$\textbf{O}_1$ and $\textbf{O}_0$ are the gates that he would need to apply
his Qbits in the standard teleporation protocol to give the state
$\ket{\psi_1}\ket{\psi_2}$, then he now needs to apply
$\textbf{S}(\textbf{O}_1\otimes\textbf{O}_0)\textbf{S}^{-1}$ to arrive at
$\textbf{S}\ket{\psi_1}\ket{\psi_0}$.\\

\begin{center}
\includegraphics[scale=.08]{/Users/AlexHahn/Desktop/awef.JPG}
\end{center}


(b) Show this works as well for Alice's initial Qbits in an arbitrary
entangled state $\ket{\Phi}$, rather than in the direct product state
$\ket{\psi_1}\ket{\psi_0}$.\\

A quick computation for a given state, say a bell state,
$\frac{1}{2}(\ket{00}+\ket{11})$ shows that we start in a bell stat, the
cNOTs flip have the time leaving us with an equiprobability
distribution. We can further extrapolate using symmetry and argue that this
will always be the case regardless of input (we will always get a bell state
output). This is because we always either apply 2 $\textbf{Z}$'s or 3
$\textbf{X}$'s which means we either always negate (-, 2 minuses), or flip
twice and end up in a bell state. ie always equiprobability
$\ket{0000},\ket{0011},\ket{1100}\ket{1111}$.\\

c) Unsure of what exactly i'm supposed to show?\\


\textbf{Problem 7}: Quantum Zeno II\\

(i) We can describe the probability of a final state of $\ket{0}=0$ by
$$R_{\pi/4}QR_{\pi/4}\ket{0}=R_{\pi/4}R_{\pi/4}\ket{0}=\ket{1}$$

For an exploding gate we have a .25 probability that the final state ends up
as $\ket{0}$. This is due to basic probability in that after the first rotation
we have .5 probability of measuring 1 (and exploding) and with a
consecutive measurement another .5 probability of getting a $\ket{0}$ or
$\ket{1}$ so an final state of $\ket{0}$ =.5$\cdot$.5=.25\\

(ii) the probability of a final state being $\ket{0}=0$:

$$(QR_{\pi/2N})^N\ket{0}=(R_{\pi/2N}^N\ket{0}=ket{1}$$

we have a 1/$N$ probability for an exploding gate; after the first $R$ we know
that there is a 1/$N^2$ chance of measuring a 1 (and exploding)
otherwise we end up back in state $\ket{0}$. Combinatorically this sequence
repeats and with $N$ times we see that $(N)(1/N^2)=1/N$ for the chance of
exploding bomb.\\


\textbf{Problem 8}: Surface code\\

Me and the two kids I'm working with (Andrew Casey and Eric Shuh) have a slight problem/
discrepency with the numbers given in the problem (and wolfram alpha does not
seem to have a stable/ reliable numerical method to solve the choose
computation). Either way, giving answers in the form of a tuple
$$(p_L,d,n_q)$$
we find
$$(10^{-10},14,729)$$
$$(10^{-20},27,2809)$$
$$10^{-30},32,3969)$$

I've checked with a couple other classmates and something seems to be off/
doesn't line up. Solving the formula for $n_q$ we get a $d$ of about 30 giving a
$p_L$ on the order of 10$^{-22}$. $n_q=3500\implies\text{error rate of
}10^{-17}$?












\end{document}
